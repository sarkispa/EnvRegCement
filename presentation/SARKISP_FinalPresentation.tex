\documentclass[12pt]{beamer}
\usepackage[utf8]{inputenc}
\usepackage[T1]{fontenc} % Font encoding: T1
\usepackage[sfdefault,scaled=.85]{FiraSans}
\usepackage{amsmath}
\usepackage{amsfonts}
\usepackage{amssymb}
\usepackage{xcolor}
\usepackage{graphicx}
\beamertemplatenavigationsymbolsempty
\setbeamertemplate{itemize items}[circle]

\newcommand{\Prob}[1]{\operatorname{Pr}\left[#1\right]}

%\setlength{\leftmargini}{10.5pt}

\AtBeginSection[]{
  \begin{frame}
  \vfill
  \centering
  \begin{beamercolorbox}[sep=8pt,center,shadow=true,rounded=true]{title}
    \usebeamerfont{title}\insertsectionhead\par%
  \end{beamercolorbox}
  \vfill
  \end{frame}
}

\newtheorem{proposition}{Proposition}
\newcommand{\E}[1]{\operatorname{E}\left[#1\right]}

\author{Paul Anthony Sarkis}
\title{The Costs of Environmental Regulation
in a Concentrated Industry \\
\textit{(Ryan, ECMA 2012)}}
%\setbeamercovered{transparent} 
%\setbeamertemplate{navigation symbols}{} 
%\logo{} 
%\institute{} 
%\date{} 
%\subject{} 
\begin{document}

\begin{frame}
\titlepage
\end{frame}

\begin{frame}{Context}
Clean Air Act in 1990 = regulation in polluting ind.\begin{itemize}
\item Increased costs of investment
\item Increased costs of entry
\item[$\Rightarrow$] Potential issue in concentrated industries
\item[]
\end{itemize}
Dynamic oligopoly analysis (vs. typical engineering estimates)
\begin{itemize}
\item Recover sunk costs, scrap values, investment costs
\item Consider dynamic effects ($\uparrow$ concentration)
\end{itemize}
\end{frame}

\begin{frame}{Methodology}
Bajari, Benkard and Levin (2007) two-step estimator:
\begin{enumerate}
\item Estimate policy functions (equilibrium behavior)
\item Recover parameters that ``justify'' policy functions
\item[]
\end{enumerate}
In this paper:
\begin{itemize}
\item Product market profits (Cournot)
\item Policy functions (state var. = capacity)
\begin{itemize}
\item Investment/Divestment
\item Entry/Exit
\end{itemize}
\item ``Dynamic part''
\end{itemize}
\end{frame}

\section{Data and Summary Statistics}

\begin{frame}{Paper dataset vs. Shared dataset}
Two datasets in use:\begin{enumerate}
\item Market-level demand/supply data
\item Plant-level production/investment data
\item[]
\end{enumerate}
First problems arise: by simple grouping, only 22 mkts (vs. 27)
\begin{itemize}
\item Some markets within a state are not ``separated''
\begin{itemize}
\item[](ex. PA represents WPA and EPA)
\item[$\Rightarrow$] + 3 markets
\end{itemize}
\item Some combinations of states are ``shuffled''
\begin{itemize}
\item[](ex. MDVAWV while MD should be alone)
\item[$\Rightarrow$] + 2 markets (but impossible to do)
\end{itemize}
\end{itemize}
\begin{center}
Bottomline: not identical dataset
\end{center}
\end{frame}


\section{Main Results}

\begin{frame}{Plan}
\begin{enumerate}
\item Product market game:
\begin{enumerate}
\item Cement demand estimation
\item Cost function estimation
\item[]
\end{enumerate}
\item Policy function estimation:
\begin{enumerate}
\item Investment/Divestment policy function
\item Entry/exit policy function
\item[]
\end{enumerate}
\item Dynamic part:
\begin{enumerate}
\item Incumbents' problem estimation
\item Entrants' problem estimation
\end{enumerate}
\end{enumerate}
\end{frame}

\begin{frame}{Cement Demand Estimation}
$$\ln Q_{jt} = \alpha_0 + \alpha_1\ln P_{jt} + \alpha_{2j} + \alpha_{3t}'X_{jt} + \epsilon_{jt} $$
Different specifications are tested:\begin{itemize}
\item Market size controls: population, construction permits
\item Market Fixed Effects
\item[$\Rightarrow$] Choose no controls but fixed effects. 
\end{itemize}
\end{frame}

\begin{frame}{Production Costs Estimation}
$$C(q_i, s_i; \delta) = \delta_1 \cdot q_i + \delta_2 \cdot 1[q_i > \nu\cdot s_i] \cdot (q_i - \nu\cdot s_i)^2 $$
Estimation using NLLS:\begin{itemize}
\item For a given guess of parameters:
\begin{itemize}
\item Solve the Cournot game in each market-year
\item Compare simulated quantities with actual (MSE)
\item Update parameters
\item[]
\end{itemize}
\item BUT objective function is highly nonlinear and takes time:
\begin{itemize}
\item[$\Rightarrow$] Global methods $>$ gradient-based
\item[$\Rightarrow$] Less function eval. $>$ More function eval.
\end{itemize}
\end{itemize}
\end{frame}

\begin{frame}{Solving the NLLS Problem}
\textit{Intuition:} Brute force until I figure out the optimal ``area'', then Powell algorithm.\\~\\
Pre-1990 period:
\begin{itemize}
\item Five rounds of brute force on a $7\times 7\times 7$ grid
\begin{itemize}
\item[] (Not a lot but already $\sim 2$ hrs...)
\end{itemize}
\item Then Powell multiple times (w/ different starting values)
\item[]
\end{itemize}
Post-1990 period:
\begin{itemize}
\item Brute force is not very effective...
\begin{itemize}
\item[] (Because obj. fun. is very flat)
\item[$\Rightarrow$] Only three rounds of BF
\end{itemize}
\item Then Powell multiple times (w/ different starting values)
\end{itemize}
\end{frame}

\begin{frame}{Investment Policy Function}
\begin{center}
(S, s) rule of investment!
\end{center}
Each firm has (1) a target capacity and (2) adjustment bands.
\begin{itemize}
\item If current cap. is within bands = no investment
\item If current cap. is outside bands = investment to target
\item[$=$] Captures lumpy investment behavior
\item[]
\end{itemize}
Based on two assumptions:
\begin{enumerate}
\item Firms adjust as soon as they hit bands = id. bands
\item Firms always adjust to target cap. = id. target
\item[$\Rightarrow$] Any observed non-zero investment tells us about both!
\end{enumerate}
\end{frame}

\begin{frame}{Investment Policy Function: Estimation}
$$\textit{(Target): } \ln(s_{it}^*) = \operatorname{Bs}\left(s_{it}, \sum_{j\neq i} s_{jt}\right) $$
$$\textit{(Bands): } \ln(s_{it}^* - s_{it}) = \operatorname{Bs}\left(s_{it}, \sum_{j\neq i} s_{jt}\right) $$
where $\operatorname{Bs}(\cdot)$ is a cubic bivariate B-spline.
\end{frame}

\begin{frame}{Entry/Exit Policy Functions}
Simple probit regression: entry/exit probability on state variables.
$$ \Prob{\textit{entry}|s_i =0, s} = \Phi\left(\psi_1 + \psi_2\cdot\left(\sum_{j\neq i} s_{jt}\right) + \psi_3\cdot 1[t \geq 1990]\right) $$
$$ \Prob{\textit{exit}|s_{it}, s} = \Phi\left(\psi_1 + \psi_2\cdot s_{it} + \psi_3\cdot\left(\sum_{j\neq i} s_{jt}\right) + \psi_4\cdot 1[t \geq 1990]\right) $$
\end{frame}

\begin{frame}{Incumbents' dynamic payoffs}
\begin{center}
Most difficult part to understand...
\end{center}
Equilibrium per-period payoff function:
\begin{align*}
\E{\pi_i(s, \sigma(s); \theta)} = \bar \pi_i(s) - & p_i(s) \cdot(\tilde \gamma_{1i} + \gamma_2 x_i + \gamma_3 x_i^2) \\ +  & p_d(s) \cdot(\tilde \gamma_{4i} + \gamma_5 x_i + \gamma_6 x_i^2) \\ + & p_e(s)\cdot \tilde{\phi}_i
\end{align*}
Normally, this function is linear in uncoditional distribution parameters! Here, where are they?
\end{frame}

\begin{frame}{What I would have done}
Consider the exit decision:
\begin{itemize}
\item $\tilde{\phi}_i$ is the expected scrap value conditional on exiting: $$\tilde{\phi}_i = \E{\phi_i | \phi > \E{\max\{V_i^+(s) - \gamma_{1i}, V_i^-(s) - \gamma_{4i}, V_i^0(s)\}}} $$
\item So given that $\phi \sim N(\mu_{\phi}, \sigma_{\phi}^2)$, then:  $$\tilde{\phi}_i = \mu_{\phi} + \sigma_{\phi} \cdot \lambda(\E{\max\{...\}}) $$
where $\lambda(\cdot)$ is the inverse Mills ratio.
\item[$\Rightarrow$] If we had $\E{\max\{...\}}$ = linear in distribution parameters!
\item[]
\end{itemize}
\begin{center}
Then, do typical BBL!
\end{center}
\end{frame}

\begin{frame}{Entrants' dynamic payoffs}
\begin{center}
Could be done if previous step was done!
\end{center}
Value function:
$$V_i^e(s; \sigma(s), \theta) = \max\{0, \max_{x_i\geq 0} -\kappa_i - \gamma_{1i} - \gamma_2x_i - \gamma_3x_i^2 + W_i(\sigma(s))\} $$
where $\kappa_i$ is entry cost draw.\\~\\

Firm will enter iff:
$$\Prob{\kappa_i + \gamma_{1i} \leq EV^e(s)} = \Phi\left( EV^e(s)\right) $$ where $\Phi = N(\mu_\kappa + \mu_\gamma^+, \sigma_\kappa^2 + \sigma_\gamma^2)$ $\Rightarrow$ solve by NLLS!
\end{frame}


\section{Discussion}

\end{document}
